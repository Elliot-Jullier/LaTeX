\documentclass[a4paper, 12pt]{scrartcl}

% Language
\usepackage[T1]{fontenc}
\usepackage{lmodern}
\usepackage[french]{babel}

% Maths
\usepackage{amsmath}
\usepackage{amssymb}
\usepackage{siunitx}

\newcommand{\cad}{c'est-à-dire}
\newcommand{\dg}{$^{\circ}$}
\newcommand{\mdg}{^{\circ}}
\renewcommand{\arraystretch}{1.4}

\begin{document}
    
\title{Electrolyse - Transformations forcées}
\date{\today}
\author{Elliot Jullier, T\dg 6}
\maketitle

\section*{I - Transformation Spontanée - Transformation Forcée}
\subsection*{1) Transformation Spontanée}

On s'interesse au système chimique formé par les couples Ox/Red : 
($\text{Cu}^{2+}$/Cu) et ($\text{Zn}^{2+}$/Zn). \\
Soit l'équation (1) : $\text{Cu}^{2+}_{(\text{aq})}\ +\ \text{Zn}_{(\text{s})} \leftrightharpoons \text{Cu}_{(\text{s})}\ +\ \text{Zn}^{2+}_{(\text{aq})}$.

\paragraph*{a) Noter vos observations \\[5mm]}

En mettant en œuvre la solution $\text{Cu}^{2+}_{\text{(aq)}}\ +\ \text{Zn}_{\text{(s)}}$, on remarque un déposé noir/ rougeâtre sur la lame de zinc, 
ce qui indique la création de $\text{Cu}_{\text{(s)}}$ \cad qu'il y a une réaction qui se produit. Le système est loin de l'équilibre.
\\[2mm]
Au contraire, dans la solution $\text{Zn}^{2+}_{\text{(aq)}}\ +\ \text{Cu}_{\text{(s)}}$ on ne remarque aucun déposé ou changement, 
on peut dire qu'aucune réaction se produit et qu'on est à l'équilibre.

\paragraph*{b) Décrire l'évolution du système chimique étudié.\\[5mm]}

On a une évolution totale dans le sens directe, on obtient :\\ 
$\text{Cu}^{2+}_{(\text{aq})}\ +\ \text{Zn}_{(\text{s})} \longrightarrow \text{Cu}_{(\text{s})}\ +\ \text{Zn}^{2+}_{(\text{aq})}$

\subsection*{2) Transformation Forcée - Electrolyse}

\paragraph*{a) Noter cos observations et les faire apparaître \\[5mm]}

La solution adopte une teinte bleuté.

\paragraph*{b) Indiquer sur le schéma le sens du courant imposé par le générateru 
et le sens de déplacement des électrons libres dans les parties métalliques du circuit \\[5mm]}

\emph{A revoir}

\paragraph*{c) Quelle est la demi-équation électronique associée à la lame de cuivre ? \\[5mm]}

$\text{Cu}_{\text{(s)}} \leftrightharpoons \text{Cu}^{2+}_{\text{(aq)}} + 2e^-$

\paragraph*{d) Quelle est la demi-équation électronique associé à la lame de zinc ? \\[5mm]}

$ \text{Zn}_{\text{(s)}} \leftrightharpoons \text{Zn}^{2+}_{\text{(aq)}} + 2e^- $

\paragraph*{e) En déduire la cathode et l'anode du circuit. \\[5mm]}

La solution prends une teinte bleuté ce qui indique la création d'un déposé de zinc, donc on a 
$\text{Zn}^{2+}_{\text{(aq)}} + 2e^- \leftrightharpoons \text{Zn}_{\text{(s)}}$ ce qui est une réduction. 
De plus, lors d'une électrolyse la réduction se produit à la cathode. On en déduit que la lame de zinc est la cathode et la lame de cuivre est l'anode.

\paragraph*{f) Ecrire léquation de la réaction qui modélise l'évolution du système \\[5mm]}

$\text{Cu}_{(\text{s})}\ +\ \text{Zn}^{2+}_{(\text{aq})} \longrightarrow \text{Cu}^{2+}_{(\text{aq})}\ +\ \text{Zn}_{(\text{s})}$

\paragraph*{g) Comparer avec la réaction précédente du 1). Conclure \\[5mm]}

Le sens d'évolution du systèmre est dans le sens indirecte car on observe la réaction inverse de la réaction 'naturelle' observée au 1), 
il s'agit donc bien d'une électrolyse.



\section*{II - Electrolyse d'une Solution d'acide Sulfurique}
\paragraph*{1) Indiquer sur le schéma le sens du courant et le sens des électrons. En déduire les électrodes qui correspondent à l'anode et la cathode. \\[5mm]}

L'électrode 1 est l'anode et l'électrode 2 est la cathode. \emph{A revoir}

\paragraph*{2) Faire l'inventaire des espèces chimiques présentes dans l'lectrolyseur sachant que l'acide sulfurique $H_2SO_4$ est un acide totalement dissocié dans l'eau. \\[5mm]}

On a $H_2O$ de l'eau, $H^+$ et $SO_4^{2-}$ de l'acide sulfurique car la dissociassion de $H_2SO_4$ donne $(2H^+; SO_4^{2-})$.

\paragraph*{3) Ecrire les demi-équations électroniques susceptibles de se produire à l'anode \\[5mm]}

\begin{itemize}
    \item[\textbullet] $2H_2O_{\text{(l)}} \leftrightharpoons 4H^+_{\text{(aq)}} + 4e^- + O_{2\text{ (g)}}$ \\[2mm]
    \item[\textbullet] $2SO_{4\text{ (aq)}}^{2-} \leftrightharpoons S_2O^{2-}_{8\text{ (aq)}}+2e^-$ \\[2mm]
\end{itemize}

\paragraph*{4) Ecrire les demi-équations susceptibles de se produire à la cathode \\[5mm]}

\begin{itemize}
    \item[\textbullet] $2H^+_{\text{(aq)}} + 2e^- \leftrightharpoons H_{2\text{ (g)}}$ \\[2mm]
    \item[\textbullet] $SO^{2-}_{4\text{ (aq)}} + 2e^-\leftrightharpoons SO_{2\text{ (g)}} + O_{2\text{ (g)}}$ \\[2mm]
\end{itemize}

\paragraph*{5) Quelles sont les observations faites ? \\[5mm]}

On observe que le volume de gaz produit à l'électrode 2 est 2 fois plus volumineux que celui crée à l'électrode 1. \\
On peut en déduire que la réaction qui se produit à l'anode est $2H_2O \leftrightharpoons 4H^+ + 4e^- + O_2$, dont le $O_2$ essaye de s'echapper. 
Et le $4H^+ + 4e^-$ fait $2\times \left[ 2H^+ +2e^- \longrightarrow H_2 \right]$ ce qui implique la production de 2 mol de $H_2$ pour chaque mol de $O_2$ 
ce qui explique la difference de volume produit. 

\paragraph*{6) Quel est le gaz formé à l'anode ? \\[5mm]}

Le gaz produit à l'anode est donc le $O_{2 \text{ (g)}}$ car le reste des produits ne sont pas des gaz.

\paragraph*{7) Identifier le gaz formé à la cathode en proposant une expérience simple. La mettre en œuvre après accord. \\[5mm]}

On suppose que le gaz crée à la cathode est du $H_2$, on vérifie cela en sortant le tube de la solution et en tenant une allumette dessous, 
on entend et observe un `bang' caractéristique de la combustion de $H_2$ et $O_2$.

\paragraph*{8) Ecrire les demi-équations qui se produisent réellement aux deux électrodes \\[5mm]}

A l'anode : $2H_2O_{\text{(l)}} \leftrightharpoons 4H^+_{\text{(aq)}} + 4e^- + O_{2\text{ (g)}}$ \\[2mm]

\indent \indent
A la cathode : $2H^+_{\text{(aq)}} + 2e^- \leftrightharpoons H_{2\text{ (g)}}$

\paragraph*{9) Ecrire l'équation de la réaction qui modélise le donctionnement de l'électrolyse. \\[5mm]}

$2H_2O_{\text{(l)}} \leftrightharpoons O_{2\text{ (g)}} + 2H_{2\text{ (g)}}$

\paragraph*{10) A l'aide d'un tableau d'avancement montrer que : $V(H_2) = 2 \times V(O_2)$. \\}

\begin{center}
\begin{tabular}{|>{\centering}p{2cm}||>{\centering}p{2cm}|>{\centering}p{2cm}|}
    \hline
    2$H_2O$ & $O_2$ & 2$H_2$ \tabularnewline
    \hline
    $2x_{max}$ & 0 & 0 \tabularnewline
    \hline
    $2x_{max}-2x$ & $x$ & $2x$ \tabularnewline
    \hline
    $0$ & $x_{max}$ & $2x_{max}$ \tabularnewline
    \hline
\end{tabular}
\end{center}

D'après le tableau d'avancement on voit que $2 \times n\left( O_2 \right) = n\left( H_2 \right)$.
\\
On assume le même volume molaire pour les deux gaz et on a $n \times V_M = V$ donc la proportionalité
tient et on a $2 \times V\left( O_2 \right) = V\left( H_2 \right)$.

\paragraph*{11) Evaluer grâce à $I$ et $\Delta t$, la quantité $Q$ mise en jeu. En déduire la quantité de matière d'électrons $n\left( e^- \right)$ ayant été 
échangés au cours de l'électrolyse. \\[5mm]}

$I = \frac{Q}{\Delta t} \iff Q = I \times \Delta t = 0,50\ \si{A} \times 300\ \si{s} = 150\ \si{C}$ \\[2mm]
De plus, $Q = F \times n\left( e^- \right) \iff n\left( e^- \right) = \frac{Q}{F} = \frac{150}{96500} = 0,001554\ \si{mol}$ d'électrons.

\paragraph*{12) A partir de l'équation de l'électrolyse exprimé le nombre d'électrons échangés en fonction de l'avancement $x$. \\[5mm]}

D'après les demi-équation de réaction mis en jeu, on sait que il y a 4 mol d'électrons transféré par mol d'étapes. Donc $x_{max} = \frac{1}{4} \times n\left( e^- \right)$.

\paragraph*{13) Exprimer le volume molaire des gaz $V_M$, en fonction de $n\left( e^- \right)$ et de $V\left( H_2 \right)$.
En déduire le volume molaire des gaz dans les conditions de l'expérience. \\[5mm]}

Pour chaque mol d'étapes, on observe la création de 2 mol de $H_2$ et par définition on a $V_M = \frac{V\left( H_2 \right)}{n\left( H_2 \right)} = \frac{V\left( H_2 \right)}{\frac{1}{2}\times n\left( H_2 \right)}$

On remplace $V_M = \frac{21\times 10^{-3}}{\frac{1}{2}\times 0,001554} = 27,02\ \si{L.mol^{-1}}$

\paragraph*{14) Retrouver la valeur de ce volume molaire en utilisant la loi des gaz parfaits et comparer avec la valeur obtenue précédemment. Calculer l'écart relatif. \\[5mm]}

D'après la loi de gaz parfaits, $PV = nRT \iff \frac{V}{n} = \frac{RT}{P}$ \\
On prends $T = 24\mdg\ \si{C}$ et $P = 101325\ \si{Pa}$ donc \\
$V_M = \frac{8,31 \times 297,15}{101325} = 0,0244 \ \si{m^3.mol^{-1}} = 24,4\ \si{L.mol^{-1}}$.
\\[2mm]
L'écart relatif est $\frac{|24,4-27,0|}{24,4} = 0,107$. \\
On peut attribuer la difference dans la valeur trouvé à l'imprécision des mesures prises tel que le volume de gaz récupéré, 
l'intensité du courant et le temps de réaction. De plus, la temperature et la pression ne sont probablement pas pareil.


\end{document}