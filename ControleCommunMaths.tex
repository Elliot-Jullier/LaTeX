\documentclass[a4paper, 12pt]{article}

% Packages

% Language
\usepackage[T1]{fontenc}
\usepackage{lmodern}
\usepackage[french]{babel}

% Maths
\usepackage{amsmath}
\usepackage{siunitx}
\usepackage{tikz}
\usetikzlibrary{shapes, calc}

% Misc
\usepackage{hyperref}
\hypersetup{
    colorlinks = true,
    urlcolor = blue!60!black,
    linkcolor = black
}
\usepackage{dsfont}

%\usepackage{showframe}

\newcommand{\tikzmark}[1]{\tikz[overlay,remember picture] \node (#1) {};}

% End Packages

\begin{document}

% Setup
\title{Correction du Contrôle Commun de Terminale Spécialités Mathématiques}
\author{Elliot Jullier}
\date{Mai 2021}
\maketitle
\pagenumbering{roman}
\tableofcontents
\newpage
\pagenumbering{arabic}
% End Setup

% Exercice 1
\phantomsection
\addcontentsline{toc}{section}{Exercice 1 (5 points)}
\section*{Exercice 1 (5 points)}

% Question 1
\phantomsection
\addcontentsline{toc}{subsection}{Question 1}
\subsection*{Question 1 :}
\noindent
On considere les droites $(d)$ et $(d')$ dont les représentations paramétriques respectives sont:  
$ \begin{cases} x = -6t + 4 \\ y = -8t - 1 \\ z = 6t - 22 \end{cases} $, $t \in \mathds{R} $ et $ \begin{cases} x = 3s + 1 \\ y = 4s \\ z = -s + 3 \end{cases} $, $s \in \mathds{R}$
\\ \\ \noindent
Ces deux droites sont :
\vspace{2mm}
\begin{itemize}
    \item[\tikzmark{bl}a)] Strictement parallèles \tikzmark{br} \vspace{1mm}
    \item[b)] Sécantes \vspace{1mm}
    \item[c)] Confondues \vspace{1mm}
    \item[d)] Non coplanaires
\end{itemize}
\tikz[overlay,remember picture]{\draw[black]
    ($(bl)+(-0.2em, 0.95em)$) rectangle
    ($(br)+(0.1em, -0.35em)$;)
}

\noindent
\underline{Justification :}
On commence par chercher si les droites $(d)$ et $(d')$ sont parallèles. 
\\
Un vecteur directeur de $(d)$ est $\vec{u} \begin{pmatrix} -6 \\ -8 \\ 6 \end{pmatrix}$ et un vecteur directeur de $(d')$ est $\vec{v} \begin{pmatrix} 3 \\ 4 \\ -3 \end{pmatrix}$.
\\
On cherche a savoir si ils sont colinéaires, c'est-à-dire si $\exists k$ tel que $ \vec{u} = k\vec{v}$ ce qui revient à faire:
\\ \\
$\begin{cases} -6 = 3k \\ -8 = 4k \\ 6 = -3k \end{cases} \Leftrightarrow \begin{cases} k = -2 \\ k = -2 \\ k = -2 \end{cases}$
\\ \\
Ce qui est vrai donc les vecteurs directeurs sont colinéaires et donc les droites sont parallèles. Il nous reste à déterminer s'il elles sont confondues ou strictement parallèles.
\\
On cherche donc a savoir si un point sur la droite $(d)$ appartient à la droite $(d')$. Un point appartenant à $(d)$ satisfait : 
$ \begin{cases} x = -6t + 4 \\ y = -8t - 1 \\ z = 6t - 22 \end{cases}$, $t \in \mathds{R}$. 
\\ \\
Si on prends $t = 0$, on a $\begin{cases} x = -6 \cdot 0 + 4 \\ y = -8 \cdot 0 - 1 \\ z = 6 \cdot 0 -22 \end{cases}$ \\
Ce qui donne les coordonnées $A(4; -1; -22)$ 
\\ \\
On veut maintenant savoir si $A(4; -1; -22) \in (d') \Leftrightarrow \begin{cases} 4 = 3s + 1 \\ -1 = 4s \\ -22 = -3s + 3 \end{cases} \Leftrightarrow \begin{cases} s = 1 \\ s = -\frac{1}{4} \\ s = \frac{25}{3} \end{cases}$  
\\ \\ 
Ce qui est impossible donc $(d)$ et $(d')$ ne sont pas confondues, ainsi il reste uniquement l'option a) Strictement parallèles.

% Question 2 
\phantomsection
\addcontentsline{toc}{subsection}{Question 2}
\subsection*{Question 2 :}
\noindent
On considère le plan $(\mathcal{P})$ dont une equation cartesienne est: $ - x + y + z - 4 = 0 $. Ce plan est parallèle à la droite $(d)$ dont une représentation paramétrique est :
\vspace{2mm}
\begin{itemize}
    \item[a)] $ \begin{cases} x = 5t + 1 \\ y = 3t \\ z = 4 \end{cases} $, $t \in \mathds{R}$
    \item[\tikzmark{bl}b)] $ \begin{cases} x = -5t + 12 \\ y = 3t \\ z = -8t \end{cases} $, $t \in \mathds{R}$ \tikzmark{br}
    \item[c)] $ \begin{cases} x = -5t - 3 \\ y = -3t \\ z = 8t \end{cases} $, $t \in \mathds{R}$
    \item[d)] $ \begin{cases} x = -t + 2 \\ y = t + 3 \\ z = t - 1 \end{cases} $, $t \in \mathds{R}$
\end{itemize}
\tikz[overlay,remember picture]{\draw[black]
    ($(bl)+(-0.2em, 2.5em)$) rectangle
    ($(br)+(0.1em, -2em)$;)
}

\noindent
\underline{Justification :}
\\
Un vecteur normal du plan $(\mathcal{P})$ d'équation cartésienne $ - x + y + z - 4 = 0$ est $\vec{n}\begin{pmatrix}-1 \\ 1 \\ 1\end{pmatrix}$.
Tout vecteur appartenant au plan est orthogonal au vecteur directeur, ainsi tout vecteur parallèle au plan est aussi orthogonal au vecteur normal de ce même plan.
De plus, deux vecteurs orthogonaux ont un produit scalaire égal à 0.
\\
On fait la liste d'un vecteur directeur possible pour la droite $(d)$ d'après la représentation parametrique proposé. 
\begin{itemize}
    \item[a)] $\begin{cases} x = 5t + 1 \\ y = 3t \\ z = 4 \end{cases}$, $t \in \mathds{R}$ a pour vecteur directeur $\vec{d_1}\begin{pmatrix} 5 \\ 3 \\ 0 \end{pmatrix}$ \\ \\ 
        $\vec{n}.\vec{d_1} = -1 \cdot 5 + 1 \cdot 3 + 1 \cdot 0 = -2 \neq 0$ \\ 
        Donc cette droite n'est pas parallèle avec $(\mathcal{P})$
    \item[b)] $\begin{cases} x = -5t +12 \\ y = 3t \\ z = -8t \end{cases}$, $t \in \mathds{R}$ a pour vecteur directeur $\vec{d_2}\begin{pmatrix} -5 \\ 3 \\ -8 \end{pmatrix}$ \\ \\ 
        $\vec{n}.\vec{d_2} = -1 \cdot (-5) + 1 \cdot 3 + 1 \cdot (-8) = 0$ \\ 
        Donc cette droite est parallèle avec $(\mathcal{P})$
    \item[c)] $\begin{cases} x = -5t - 3 \\ y = -3t \\ z = 8t \end{cases}$, $t \in \mathds{R}$ a pour vecteur directeur $\vec{d_3}\begin{pmatrix} -5 \\ -3 \\ 8 \end{pmatrix}$ \\ \\ 
        $\vec{n}.\vec{d_3} = -1 \cdot (-5) + 1 \cdot (-3) + 1 \cdot 8 = 10 \neq 0$ \\ 
        Donc cette droite n'est pas parallèle avec $(\mathcal{P})$
    \item[d)] $\begin{cases} x = -t + 2 \\ y = t + 3 \\ z = t - 1 \end{cases}$, $t \in \mathds{R}$ a pour vecteur directeur $\vec{d_4}\begin{pmatrix} -1 \\ 1 \\ 1 \end{pmatrix}$ \\ \\ 
    Le vecteur directeur de cette droite est colinéaire avec le vecteur normal du plan, ainsi cette droite est orthogonale au plan et donc ne peut être parallèle à $(\mathcal{P})$\\ 
\end{itemize}
Donc il y a que la deuxième droite dont une représentation paramétrique est $\begin{cases} x = -5t + 12 \\ y = 3t \\ z = 8-t \end{cases}$ 
qui est parallèle au plan $(\mathcal{P})$ d'équation cartésienne $ - x + y + z - 4 = 0$.

% Question 3
\phantomsection
\addcontentsline{toc}{subsection}{Question 3}
\subsection*{Question 3 :}
\noindent
On considère la droite $(\mathcal{D})$ passant par $A(-1; 2; 3)$ et de vecteur directeur $\vec{u}\begin{pmatrix}4 \\ -2 \\ 1 \end{pmatrix}$. \\
Une représentation de $(\mathcal{D})$ (avec $t \in \mathds{R}$) est :
\vspace{2mm}
\begin{itemize}
    \item[a)] $\begin{cases} x = -1 + 4t \\ y = 2 + 2t \\ z = 3 + t \end{cases}$
    \item[b)] $\begin{cases} x = 4t + 1 \\ y = -2t + 2 \\ z = t + 3 \end{cases}$
    \item[\tikzmark{bl}c)] $\begin{cases} x = -5 - 6t \\ y = 4 + 3t \\ z = 2 -  1,5t \end{cases}$ \tikzmark{br}
    \item[d)] $\begin{cases} x = -2 + 8t \\ y = 4 - 4t \\ z = 6 + 2t \end{cases}$   
\end{itemize}
\tikz[overlay,remember picture]{\draw[black]
    ($(bl)+(-0.2em, 2.5em)$) rectangle
    ($(br)+(0.1em, -2em)$;)
}

\noindent
\underline{Justification :}
\\
On commence par chercher quelles représentations paramétriques ont un vecteur directeur colinéaire avec le vecteur $\vec{u}$. S'il ne le sont pas alors la représentation paramétrique 
ne peut être une représentation de $(\mathcal{D})$. 
\vspace{3mm}
\begin{itemize}
    \item[a)] $\begin{cases} x = -1 + 4t \\ y = 2 + 2t \\ z = 3 + t \end{cases}$ a un vecteur directeur $\vec{d_1}\begin{pmatrix} 4 \\ 2 \\ 1 \end{pmatrix}$. 
    \\On cherche a savoir s'il existe une solution a $\begin{cases} 4k = 4 \\ 2k = -2 \\ k = 1 \end{cases} \Leftrightarrow \begin{cases} k = 1 \\ k = -1 \\k = 1 \end{cases}$ 
    \\Ce qui est impossible, donc ça ne peut être a). \vspace{5mm}
    \item[b)]  $\begin{cases} x = 4t + 1 \\ y = -2t + 2 \\ z = t + 3 \end{cases}$ a un vecteur directeur $\vec{d_2}\begin{pmatrix} 4 \\ -2 \\ 1 \end{pmatrix}$. 
    \\On cherche a savoir s'il existe une solution a $\begin{cases} 4k = 4 \\ -2k = -2 \\ k = 1 \end{cases} \Leftrightarrow \begin{cases} k = 1 \\ k = 1 \\k = 1 \end{cases}$ 
    \\Donc b) pourrait être solution. \vspace{5mm}
    \item[c)] $\begin{cases} x = -5 - 6t \\ y = 4 + 3t \\ z = 2-1,5t \end{cases}$ a un vecteur directeur $\vec{d_3}\begin{pmatrix} -6 \\ 3 \\ -1,5 \end{pmatrix}$. 
    \\On cherche a savoir s'il existe une solution a $\begin{cases} -6k = 4 \\ 3k = -2 \\ -1,5k = 1 \end{cases} \Leftrightarrow \begin{cases} k = -\frac{2}{3} \\ k = -\frac{2}{3} \\k = -\frac{2}{3} \end{cases}$ 
    \\Donc c) pourrait être solution. \vspace{5mm}
    \item[d)] $\begin{cases} x = -2 - 8t \\ y = 4 - 4t \\ z = 6 + 2t \end{cases}$ a un vecteur directeur $\vec{d_4}\begin{pmatrix} 8 \\ -4 \\ 2 \end{pmatrix}$. 
    \\On cherche a savoir s'il existe une solution a $\begin{cases} 8k = 4 \\-4k = -2 \\ 2k = 1 \end{cases} \Leftrightarrow \begin{cases} k = \frac{1}{2} \\ k = \frac{1}{2} \\k = \frac{1}{2} \end{cases}$
    \\Donc d) pourrait être solution. \vspace{5mm}
\end{itemize}

Il faut maintenant vérifier pour lequel d'entre eux le point A appartient à la droite :
\vspace{3mm}
\begin{itemize}
    \item[b)] $A(-1; 2; 3) \in \begin{cases} x = 4t + 1 \\ y = -2t + 2 \\ z = t + 3 \end{cases} 
        \Leftrightarrow \begin{cases} -1 = 4t + 1\\ 2 = -2t + 2\\ 3 = t + 3\end{cases} 
        \Leftrightarrow \begin{cases} t = -\frac{1}{2} \\ t = 0 \\ t = 0\end{cases}$
        \\ Donc b) n'est pas solution. \vspace{3mm}
    \item[c)] $A(-1; 2; 3) \in \begin{cases} x = -5 - 6t \\ y = 4 + 3t \\ z = 2 - 1,5t \end{cases} 
        \Leftrightarrow \begin{cases} -1 = -5 -6t\\ 2 = 4 + 3t \\ 3 = 2 - 1,5t\end{cases} 
        \Leftrightarrow \begin{cases} t = -\frac{2}{3} \\ t = -\frac{2}{3} \\ t = - \frac{2}{3} \end{cases}$
        \\ Donc c) est solution. \vspace{3mm}
    \item[d)] $A(-1; 2; 3) \in \begin{cases} x = -2 + 8t \\ y = 4 - 4t \\ z = 6 + 2t \end{cases} 
        \Leftrightarrow \begin{cases} -1 = -2 +8t \\ 2 = 4 - 4t\\ 3 = 6 + 2t\end{cases} 
        \Leftrightarrow \begin{cases} t = \frac{1}{8} \\ t = -\frac{1}{2} \\ t = -\frac{3}{2} \end{cases}$
        \\ Donc d) n'est pas solution.
\end{itemize}
\vspace{3mm}
Donc c) est une représentation de la droite $(\mathcal{D})$.

% Question 4
\phantomsection
\addcontentsline{toc}{subsection}{Question 4}
\subsection*{Question 4 :}
\noindent
Soient $A(1; \frac{1}{2}; 1)$, $B(-1; 1; 2)$ et $C(0; 0; 3)$ trois points de l'espace. \\
Lequel des vecteurs suivants est normal au plan $(ABC)$ ? 
\vspace{3mm}
\begin{itemize}
    \item[a)] $\begin{pmatrix} 0 \\ -\frac{1}{2} \\ 1 \end{pmatrix}$ \vspace{2mm}
    \item[b)] $\begin{pmatrix} -1 \\ -\frac{1}{2} \\ 2 \end{pmatrix}$ \vspace{2mm}
    \item[c)] $\begin{pmatrix} 3 \\ -2 \\ 1 \end{pmatrix}$ \vspace{2mm}
    \item[\tikzmark{bl}d)] $\begin{pmatrix} 1 \\ 2 \\ 1 \end{pmatrix}$  \tikzmark{br} 
\end{itemize}
\tikz[overlay,remember picture]{\draw[black]
    ($(bl)+(-0.2em, 2.5em)$) rectangle
    ($(br)+(0.1em, -2em)$;)
}
\vspace{3mm}

\noindent
\underline{Justification :}
\\
Tout vecteur normal au plan est orthogonal aux vecteurs appertenant au plan. 
C'est a dire que le produit scalaire entre le vecteur normal et deux vecteurs non-colinéaires appartenant au plan doivent tout les deux être égal à 0.
\\
On a $\overrightarrow{\text{AB}}\begin{pmatrix}-1-1\\1-\frac{1}{2} \\ 2-1\end{pmatrix} = \overrightarrow{\text{AB}}\begin{pmatrix} -2 \\ \frac{1}{2} \\ 1\end{pmatrix}$ et 
$\overrightarrow{\text{AC}}\begin{pmatrix} 0-1 \\ 0- \frac{1}{2} \\ 3-1 \end{pmatrix} = \overrightarrow{\text{AB}}\begin{pmatrix} -1 \\ -\frac{1}{2} \\ 2 \end{pmatrix}$
\\
On vérifie que $\overrightarrow{\text{AB}} \neq k\overrightarrow{\text{AC}}$, $\forall k \in \mathds{R}$.
\\
$\begin{cases} -2 = -k \\ \frac{1}{2} = -\frac{1}{2}k \\ 1 = 2k \end{cases} \Leftrightarrow \begin{cases}k = 2 \\ k = -1 \\ k = \frac{1}{2} \end{cases}$ 
\\
Cela est impossible donc $\overrightarrow{\text{AB}}$ et $\overrightarrow{\text{AC}}$ ne sont pas colinéaires et donc sont des vecteurs directeurs du plan. 
\\
On vérifie maintenant pour chaque vecteur s'il est orthogonal aux deux autres :
\vspace{3mm}
\begin{itemize}
    \item[a)] $\begin{pmatrix} 0 \\ -\frac{1}{2} \\ 1 \end{pmatrix} .\ \overrightarrow{\text{AB}} = 0 \cdot (-2) + (-\frac{1}{2}) \cdot \frac{1}{2} + 1 \cdot 1 = -\frac{3}{4} \neq 0$
        \\ Donc $\begin{pmatrix} 0 \\ -\frac{1}{2} \\ 1 \end{pmatrix}$  et $\overrightarrow{\text{AB}}$ ne sont pas orthogonaux donc a) n'est pas solution. \vspace{2mm}
    \item[b)] $\begin{pmatrix} -1 \\ -\frac{1}{2} \\ 2 \end{pmatrix} .\ \overrightarrow{\text{AB}} = -1 \cdot (-2) + (-\frac{1}{2}) \cdot \frac{1}{2} + 2 \cdot 1 = \frac{11}{4} \neq 0$
        \\ Donc $\begin{pmatrix} -1 \\ -\frac{1}{2} \\ 2 \end{pmatrix}$  et $\overrightarrow{\text{AB}}$ ne sont pas orthogonaux donc b) n'est pas solution. \vspace{2mm}
    \item[c)] $\begin{pmatrix} 3 \\ -2 \\ 1 \end{pmatrix} .\ \overrightarrow{\text{AB}} = 3 \cdot (-2) + (-2) \cdot \frac{1}{2} + 1 \cdot 1 = -6 \neq 0$
        \\ Donc $\begin{pmatrix} 3 \\ -2 \\ 1 \end{pmatrix}$  et $\overrightarrow{\text{AB}}$ ne sont pas orthogonaux donc c) n'est pas solution. \vspace{2mm}
    \item[d)] $\begin{pmatrix} 1 \\ 2 \\ 1 \end{pmatrix} .\ \overrightarrow{\text{AB}} = 1 \cdot (-2) + 2 \cdot \frac{1}{2} + 1 \cdot 1 = 0$ et
        \\ $\begin{pmatrix} 1 \\ 2 \\ 1 \end{pmatrix} .\ \overrightarrow{\text{AC}} = 1 \cdot (-1) + 2 \cdot (-\frac{1}{2}) + 1 \cdot 2 = 0$
        \\ Donc  $\begin{pmatrix} 1 \\ 2 \\ 1 \end{pmatrix}$ est orthogonal à $\overrightarrow{\text{AB}}$ et à $\overrightarrow{\text{AC}}$, alors il est normal au plan $(\mathcal{P})$.
\end{itemize}
\vspace{3mm}
Donc d) est la solution

% Question 5
% Check answer box not working
\phantomsection
\addcontentsline{toc}{subsection}{Question 5}
\subsection*{Question 5 :}
\noindent
Soient $E(2; 1; -3)$, $F(1; -1; 2)$ et $G(-1; 3; 1)$ trois points de l'espace.
\\
Une mesure arrondie au degré de l'angle $\widehat{\text{FED}}$ est:
\begin{itemize}
    \item[a)] $48^{\circ}$
    \item[b)] $49^{\circ}$
    \item[\tikzmark{bl}c)] $50^{\circ}$ \tikzmark{br}
    \item[d)] $51^{\circ}$   
\end{itemize}
\tikz[overlay,remember picture]{\draw[black]
    ($(bl)+(-0.2em, 0.9em)$) rectangle
    ($(br)+(0.05em, -0.35em)$;)
}
\vspace{3mm}
\vspace{2cm}

\noindent
\underline{Justification :}
\\ \\
On a $\overrightarrow{\text{EF}}\begin{pmatrix} 1-2 \\ -1-1 \\2-(-3)\end{pmatrix} = \overrightarrow{\text{EF}}\begin{pmatrix} -1 \\ -2 \\ 5 \end{pmatrix}$ 
et $\overrightarrow{\text{EG}}\begin{pmatrix} -1-2 \\ 3-1\\1-(-3)\end{pmatrix} = \overrightarrow{\text{EG}}\begin{pmatrix} -3 \\ 2 \\ 4 \end{pmatrix}$
\\
On calcule $\overrightarrow{\text{EF}} .\ \overrightarrow{\text{EG}} = -1 \cdot (-3) + (-2) \cdot 2 + 5 \cdot 4 = 19$
\\
De plus, $\overrightarrow{\text{EF}} .\ \overrightarrow{\text{EG}} = \text{EF} \cdot \text{EG} \cdot \cos{\left( \widehat{FEG} \right)}$, 
$\text{EF} = \|\overrightarrow{\text{EF}}\| = \sqrt{(-1)^2 + (-2)^2 + 5^2} = \sqrt{30}$ et 
$\text{EG} = \| \overrightarrow{\text{EG}} \| = \sqrt{(-3)^2 + 2^2 + 4^2} = \sqrt{29}$.
\\
Donc $\overrightarrow{\text{EF}} .\ \overrightarrow{\text{EG}} = \sqrt{30} \cdot \sqrt{29} \cdot \cos{\left( \widehat{FEG} \right)} = 19 \Leftrightarrow \cos{\left( \widehat{FEG} \right)} = \frac{19}{\sqrt{30} \cdot \sqrt{29}}$
\\
Ainsi, $\widehat{\text{FEG}} = \cos^{-1}{ \left( \frac{19}{ \sqrt{30} \cdot \sqrt{29}} \right) } \approx 49,897^{\circ}$ on arrondi à l'unité ce qui donne $50^{\circ}$.

% Exercice 2
\phantomsection
\addcontentsline{toc}{section}{Exercice 2 (5 points)}
\section*{Exercice 2 (5 points)}

Les trois parties A, B et C sont indépendantes. On arrondira les résultats à $10^{-3}$.

% Partie A
\phantomsection
\addcontentsline{toc}{subsection}{Partie A}
\subsection*{Partie A - }
Le premier ministre (PM) d'une certaine île doit prendre tous les jours des décisions importantes pour son pays. Lorsqu'il est fatigué, il tire ô pile ou face avec une pièce équilibrée.
Quand il n'est pas fatigué, on estme qu'il prend 40\% de bonnes décisions. Malheureusement pour ses concitoyens, il n'est que fatigué qu'un tiers du temps. On note \emph{\textbf{F}} l'événement 
\guillemotleft \ Le premier ministre est fatigué \guillemotright \ et \emph{\textbf{B}} l'événement \guillemotleft \ Il prend une bonne décision \guillemotright.

% Question 1
\phantomsection
\addcontentsline{toc}{subsubsection}{Question 1}
\subsubsection*{Question 1 :}
\paragraph*{Calculer la probabilité que le premier ministre prenne une bonne décision. On pourra représenter la situation par un arbe de probabilité.\\[5mm]}

% Insert TikZ arbre pondere
$\emph{\textbf{F}}$ et $\overline{\emph{\textbf{F}}}$ forment une partition de l'univers, donc d'après la formule des probabilités totales 
$p(\emph{\textbf{B}}) = p(\emph{\textbf{F}}\  \cap \ \emph{\textbf{B}}) + p(\overline{\emph{\textbf{F}}}\ \cap \ \emph{\textbf{B}}) 
= p(\emph{\textbf{F}}) \cdot p_{\emph{\textbf{F}}}(\emph{\textbf{B}}) + p(\overline{\emph{\textbf{F}}}) \cdot p_{\overline{\emph{\textbf{F}}}}(\emph{\textbf{B}}) 
= \frac{1}{3} \cdot 0,5 + \frac{2}{3} \cdot 0,4 = \frac{13}{30} \approx 0,433$
\\
La probabilité que le PM prenne une bonne décision est d'environ 0,433.

% Question 2
\phantomsection
\addcontentsline{toc}{subsubsection}{Question 2}
\subsubsection*{Question 2 :}
\paragraph*{On annonce ce matin encore une mauvaise décision du premier ministre, quelle est la probabilité qu'il était fatigué lorsqu'il l'a prise ?\\[5mm]}

$p_{\overline{\emph{\textbf{B}}}}(\emph{\textbf{F}}) = \frac{p(\emph{\textbf{F}}\ \cap \ \overline{\emph{\textbf{B}}})}{p(\overline{\emph{\textbf{B}}})}
= \frac{p(\emph{\textbf{F}})\ \cdot \ p_{\emph{\textbf{F}}}(\overline{\emph{\textbf{B}}})}{p(\overline{\emph{\textbf{B}}})}
= \frac{\frac{1}{3}\ \cdot \ 0,5}{1-p(\emph{\textbf{B}})} = \frac{5}{17} \approx 0,294$.
\vspace{2cm}

% Partie B
\phantomsection
\addcontentsline{toc}{subsection}{Partie B}
\subsection*{Partie B -}
Les jours se succedènt et se ressemblent. On admet dans cette partie que le PM prend une décision par jour et que la décision de la veille n'a pas d'impact sur la suivante. 
On admet également que la probabilité qu'il prenne une bonne décision est de $\frac{13}{30}$.
\\
On note $X$ la variable aléatoire comptant le nombre de bonnes décisions prises par le PM.

% Question 1 
\phantomsection
\addcontentsline{toc}{subsubsection}{Question 1}
\subsubsection*{Question 1 :}
\paragraph*{Quelle loi suit la variable aléatoire $X$ ? Justifier. \\[5mm]}

On répète plusieurs fois de manière indépendante une épreuve de Bernoulli don le succès $X$ est l'événement \guillemotleft \ Le PM a pris une bonne décision\ \guillemotright \
qui a pour probabilité $p = \frac{13}{30}$. La variable aléatoire $X$ suit donc une loi binomiale.

% Question 2 
\phantomsection
\addcontentsline{toc}{subsubsection}{Question 2}
\subsubsection*{Question 2 :}
\paragraph*{Une semaine vient de se passer (5 jours). Quelle est la probabilité qu'au moins trois bonnes décisions aient été prises ? Justifier.\\[5mm]}

Prendre au moins 3 bonnes décisions implique que il n'a pas pris moins de 2 bonnes décisions.\\
$p(X \geq 3) = 1 - p(x \leq 2) = 1- \left( p(X=0) + p(X = 1) + p(X = 2)\right) \approx 0,624$

% Question 3 
\phantomsection
\addcontentsline{toc}{subsubsection}{Question 3}
\subsubsection*{Question 3 :}
\paragraph*{Au bout de combien de jours la probabilité qu'au moins une bonne décision soit prise sera supérieure à 0,99 ? On détaillera la démarche.\\[5mm]}

Prendre au moins une bonne décision sur $n$ jours revient a ne pas prendre de mauvaise décisions sur $n$ jours. Donc : \\
$\begin{aligned}
p(X \geq 1) \geq 0,99 &\Leftrightarrow 1 - p(X = 0) \geq 0,99 \\
&\Leftrightarrow p(X = 0) \leq 0,01 \\
&\Leftrightarrow \begin{pmatrix} n \\ 0 \end{pmatrix} \left( \frac{13}{30}\right)^0 \left(1 - \frac{13}{30}\right)^{n-0} \leq 0,01 \\
&\Leftrightarrow 1 \cdot 1 \cdot \left(\frac{17}{30} \right)^n \leq 0,01 \\
&\Leftrightarrow n \cdot \ln{\left( \frac{17}{30} \right) } \leq \ln{\left(0,01\right)} \text{ car } x \mapsto \ln{\left(x\right)} \text{ est croissante sur } \mathds{R}^+_*\\
&\Leftrightarrow n  \geq \frac{\ln{\left(0,01\right)}}{\ln{\left(17\right)}-\ln{\left(30\right)}} \text{ car } 0 \leq \frac{17}{30} \leq 1 
    \text{ donc } \ln{\left(\frac{17}{30}\right)} \leq 0 \\
& \text{ donc il faut changer de sens.}
\end{aligned}$
\\[2mm]
Alors, $n \geq \frac{\ln{\left(0,01\right)}}{\ln{\left(17\right)}\ -\ \ln{\left(30\right)}} \approx 8,108 $
\\
Il faut qu'au moins 9 jours se soit écoulé pour que la probabilité que le PM ait pris au moins une bonne décision soit supérieure à 0,99.

% Partie C
\phantomsection
\addcontentsline{toc}{subsection}{Partie C}
\subsection*{Partie C -}
Pour renflouer les caisses de l'état, le PM décide de vendre des billets d'un jeu de grattage \guillemotleft\ Save Britain\ \guillemotright. Chaque ticket comporte deux cases à gratter. 
Les gains des deux cases sont donnés par : 
\\

\begin{center}
\begin{tabular}{ |l|c|c|c| }
    \hline
    Gain case 1 & £0 & £2 & £5 \\
    \hline
    Fréquence & 70\% & 20\% & 10\% \\
    \hline
\end{tabular}
\hspace{1cm}
\begin{tabular}{ |l|c|c|c| }
    \hline
    Gain case 2 & £0 & £5 & £10 \\
    \hline
    Fréquence & 70\% & 25\% & 5\% \\
    \hline
\end{tabular}
\end{center}

% Question 1 
\phantomsection
\addcontentsline{toc}{subsubsection}{Question 1}
\subsubsection*{Question 1 :}
\paragraph*{Calculer l'espérance de gain de chacune des deux cases, puis l'espérance de gain d'un ticket.\\[5mm]}

On note $T_1$ la variable aléatoire comptant le gain provenant de la case 1 et $T_2$ la variable aléatoire comptant le gain provenant de la case 2, 
et $G = T_1 + T_2$ les gains (pour le joueur) de chaque ticket.
\\
$\text{E}(T_1) = 0,70 \cdot 0 + 0,20 \cdot 2 + 0,10 \cdot 5 = \pounds 0,9$ \\
$\text{E}(T_2) = 0,70 \cdot 0 + 0,25 \cdot 5 + 0,05 \cdot 10 = \pounds 1,75$ \\
Donc $\text{E}(G) = \text{E}(T_1 + T_2) = \text{E}(T_1) + \text{E}(T_2)$ car $T_1$ et $T_2$ sont indépendant l'un de l'autre et $\text{E}(G) = 0,9 + 1,75 = \pounds 2,65$

% Question 2 
\phantomsection
\addcontentsline{toc}{subsubsection}{Question 2}
\subsubsection*{Question 2 :}
\paragraph*{A quel prix doit être vendu un ticket pour que l'espérance de gain \underline{pour l'état} soit de £850 pour 1000 tickets vendus ?\\[5mm]}

Si le gain pour un joueur d'un ticket a une espérance de £2,65 alors 1000 tickets ont une espérance de $\text{E}(1000G) = 1000\cdot \text{E}(G) = \pounds 2650$. 
Les gains de l'état peut être décrit comme la difference entre le bénéfice initial généré par la vente des tickets et l'espérance des gains pour les joueurs ce qui est une perte pour l'état.
On note $P$ le prix du ticket. \\
On a donc : \\
$\pounds 850 = 1000 \cdot P - \pounds 2650 \Leftrightarrow P = \frac{2650\ +\ 850}{1000} = \pounds 3,50$ \\
L'état doit vendre chaque ticket à £3,50 pour qu'en moyenne il tire un gain de £850 pour 1000 tickets de vendu.
\vspace{1cm}

% Exercice 3
\phantomsection
\addcontentsline{toc}{section}{Exercice 3 (6 points)}
\section*{Exercice 3 (6 points)}
\end{document}

% Notes 

% https://tex.stackexchange.com/questions/33443/box-around-single-element-in-list